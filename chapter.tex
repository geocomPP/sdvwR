\documentclass[]{article}
\usepackage[T1]{fontenc}
\usepackage{lmodern}
\usepackage{amssymb,amsmath}
\usepackage{ifxetex,ifluatex}
\usepackage{fixltx2e} % provides \textsubscript
% use upquote if available, for straight quotes in verbatim environments
\IfFileExists{upquote.sty}{\usepackage{upquote}}{}
\ifnum 0\ifxetex 1\fi\ifluatex 1\fi=0 % if pdftex
  \usepackage[utf8]{inputenc}
\else % if luatex or xelatex
  \usepackage{fontspec}
  \ifxetex
    \usepackage{xltxtra,xunicode}
  \fi
  \defaultfontfeatures{Mapping=tex-text,Scale=MatchLowercase}
  \newcommand{\euro}{€}
\fi
% use microtype if available
\IfFileExists{microtype.sty}{\usepackage{microtype}}{}
\usepackage{color}
\usepackage{fancyvrb}
\DefineShortVerb[commandchars=\\\{\}]{\|}
\DefineVerbatimEnvironment{Highlighting}{Verbatim}{commandchars=\\\{\}}
% Add ',fontsize=\small' for more characters per line
\newenvironment{Shaded}{}{}
\newcommand{\KeywordTok}[1]{\textcolor[rgb]{0.00,0.44,0.13}{\textbf{{#1}}}}
\newcommand{\DataTypeTok}[1]{\textcolor[rgb]{0.56,0.13,0.00}{{#1}}}
\newcommand{\DecValTok}[1]{\textcolor[rgb]{0.25,0.63,0.44}{{#1}}}
\newcommand{\BaseNTok}[1]{\textcolor[rgb]{0.25,0.63,0.44}{{#1}}}
\newcommand{\FloatTok}[1]{\textcolor[rgb]{0.25,0.63,0.44}{{#1}}}
\newcommand{\CharTok}[1]{\textcolor[rgb]{0.25,0.44,0.63}{{#1}}}
\newcommand{\StringTok}[1]{\textcolor[rgb]{0.25,0.44,0.63}{{#1}}}
\newcommand{\CommentTok}[1]{\textcolor[rgb]{0.38,0.63,0.69}{\textit{{#1}}}}
\newcommand{\OtherTok}[1]{\textcolor[rgb]{0.00,0.44,0.13}{{#1}}}
\newcommand{\AlertTok}[1]{\textcolor[rgb]{1.00,0.00,0.00}{\textbf{{#1}}}}
\newcommand{\FunctionTok}[1]{\textcolor[rgb]{0.02,0.16,0.49}{{#1}}}
\newcommand{\RegionMarkerTok}[1]{{#1}}
\newcommand{\ErrorTok}[1]{\textcolor[rgb]{1.00,0.00,0.00}{\textbf{{#1}}}}
\newcommand{\NormalTok}[1]{{#1}}
\usepackage{graphicx}
% We will generate all images so they have a width \maxwidth. This means
% that they will get their normal width if they fit onto the page, but
% are scaled down if they would overflow the margins.
\makeatletter
\def\maxwidth{\ifdim\Gin@nat@width>\linewidth\linewidth
\else\Gin@nat@width\fi}
\makeatother
\let\Oldincludegraphics\includegraphics
\renewcommand{\includegraphics}[1]{\Oldincludegraphics[width=\maxwidth]{#1}}
\ifxetex
  \usepackage[setpagesize=false, % page size defined by xetex
              unicode=false, % unicode breaks when used with xetex
              xetex]{hyperref}
\else
  \usepackage[unicode=true]{hyperref}
\fi
\hypersetup{breaklinks=true,
            bookmarks=true,
            pdfauthor={},
            pdftitle={},
            colorlinks=true,
            urlcolor=blue,
            linkcolor=magenta,
            pdfborder={0 0 0}}
\urlstyle{same}  % don't use monospace font for urls
\setlength{\parindent}{0pt}
\setlength{\parskip}{6pt plus 2pt minus 1pt}
\setlength{\emergencystretch}{3em}  % prevent overfull lines
\setcounter{secnumdepth}{0}

\author{}
\date{}

\begin{document}

\section{Introduction}

\subsection{What is R?}

R is a free and open source computer program that runs on all major
operating systems. R relies primarily on a \emph{command line} interface
for data input: instead of interacting with the program by moving your
mouse around clicking on different parts of the screen, users enter
commands via the keyboard. This will seem to strange to people
accustomed to relying on a graphical user interface (GUI) for most of
their computing, e.g.~via popular programs such as Microsoft Excel or
SPSS, yet the approach has a number of benefits, as highlighted by Gary
Sherman (2008, p.~283), developer of the popular GIS program QGIS:

\begin{quote}
With the advent of ``modern'' GIS software, most people want to point
and click their way through life. That's good, but there is a tremendous
amount of flexibility and power waiting for you with the command line.
Many times you can do something on the command line in a fraction of the
time you can do it with a GUI.
\end{quote}

The joy of this, when you get accustomed to it, is that any command is
only ever a few keystrokes away, and the order of the commands sent to R
can be stored and repeated in scripts, saving even more time in the
long-term (more on this in section \ldots{}).

Another important attribute of R, related to its command line interface,
is that it is a fully fledged \emph{programming language}. Other GIS
programs are written in lower level languages such as C++ which are kept
at a safe distance from the users by the GUI. In R, by contrast, the
user is `close to the metal' in the sense that what he or she inputs is
the same as what R sees when it processes the request. This `openness'
can seem raw and daunting to beginners, but it is vital to R's success.
Access to R's source code and openness about how it works has enabled a
veritable army of programmers to improve R over time and add an
incredible number of extensions to its base capabilities. Consider for a
moment that there are now more than 4000 official packages for R,
allowing it to tackle almost any computational or numerical problem one
could image, and many more that one could not!

Although writing R source code and creating new packages will not appeal
to most R users, it inspires confidence to know that there is a strong
and highly skilled community of R developers. If there is a useful
spatial function that R cannot currently perform, there is a reasonable
chance that someone is working on a solution that will become available
at a later date. This constant evolution and improvement is a feature of
open source software projects not limited to R, but the range and
diversity of extensions is certainly one of its strong points. One area
where extension of R's basic capabilities has been particularly
successful is the addition of a wide variety of spatial tools.

\subsection{The rise of R's spatial capabilities}

!!! Quick history of R's spatial packages emphasizing current growth and
heavy dependence on sp.

Mention exciting and recently added packages.

\subsection{Why R for spatial data visualisation?}

Aside from confusion surrounding its one character name - ``what kind of
a name is R?'' {[}1{]} and ``how can you possibly find resources for R
online?'' {[}2{]} - R may also seem a strange choice for a chapter on
\emph{spatial} data visualisation specifically. ``I thought R was just
for statistics?'' and ``Why not use a proper GIS package like QGIS?''
are valid questions.

The first question arises because R was traditionally conceived - and is
still primarily known - as a ``statistical programming language''
(Bivand and Gebhardt 2000). Although R does have cutting edge
statistical capabilities, this definition does not do justice to its
power and flexibility. Thus, a more accurate albeit longer definition of
R is ``an integrated suite of software facilities for data manipulation,
calculation and graphical display'' (Venables et al. 2013). It is
important to consider this wider definition before diving into R: it is
a fully fledged programming language meaning that it is highly
extensible but also that the same result can often be generated in
different ways. This can be confusing.

The second question is based on the premise that all `proper' Geographic
Information Systems need to operate in the same way, with primacy
allocated to a mapping window and a mouse-driven GUI interface. But when
we look back at the history of GIS and its definitions, it becomes clear
that R \emph{is} fully fledged GIS, when it is set up correctly. All
early GIS programs used a command-line interface; GUIs were only
developed later as a way to run commands without needing to remember all
the command names (although this is largely overcome by good `help'
options and auto-completion). A concise definition of a GIS is ``a
computerized tool for solving geographic problems'' (Longley et al.
2005, p.~16) and R certainly enables this. A more expansive definition
of GIS is ``a powerful set of tools for collecting, storing, retrieving
at will, transforming, and displaying spatial data from the real world
for a particular set of purposes'' (Burrough and McDonnell, 1998, from
Bivand et al. 2013, p.~5); R excels at each of these tasks.

That being said, there are a few major differences between R and
conventional GIS programs in terms of spatial data visualisation: R is
more suited to creating one-off graphics than exploring spatial data
interactively on a map. Conventional GIS packages are better at repeated
zooming, panning and spatial sub-setting using custom-drawn polygons
than R. Use of the \texttt{locator} function allows some interactive
selection capabilities in R, but these are limited (Bivand et al. 2013,
3.4). Although interactive maps in R can be created (e.g.~using the web
interface \texttt{shiny}), R should not be seen as a direct replacement
of dedicated GIS programs, especially now that there are myriad free
options to try (Sherman 2008). One should use the program which is most
appropriate for the task: R can tackle almost any spatial visualisation
problem and may be the best option in many cases. In others, however, it
may be best used alongside other programs (e.g.~Google Earth).

While dedicated GIS programs handle spatial data by default and display
the results in a single way, there are various options in R that must be
decided by the user. This can be daunting. For example, the user must
decide whether to use R's base graphics or a dedicated graphics package
such as ggplot2 for mapping. On the other hand, a major benefit of R is
that allows spatial and non-spatial analysis to occur in a
\emph{consistent} and \emph{cohesive} framework. Another benefit of R
for spatial data visualisation lies in the \emph{reproducibility} of its
outputs, a feature that we will be using to great effect in this
chapter.

\subsection{R for Reproducible research}

!!! Are all the examples going to be reproducible?

All these components - scripting, stability and the ability to embed
`live' code in documents - make R an excellent tool for transparent
research. By using R and carefully documenting what has been done, one
ensures that the methods used to reach a certain result can be
reproduced by anyone anywhere in the world, provided they have access to
the input dataset. The RStudio graphical interface with R encourages
good documentation. RStudio enables presentations to be created and
professional-quality pdf documents to be produced using the custom file
formats \texttt{.Rpres} and \texttt{.rnw}. In fact, this chapter was
written in RMarkdown and converted into a pdf document using only
RStudio editor!

\subsection{R in the wild}

Examples of where R has had an important visual impact.

Might be good to mention New York Times etc here as key users of R.

\subsection{An introductory session}

The best way to learn to use a new tool is by using it. The metaphor of
craft skills is appropriate here: if you wanted to become skilled at
scything, for example, you would not spend your time reading about
scythes. The same is true of R: the best way to learn how it works is to
`get your hands dirty' and try it out on your own computer. This
introductory session will therefore serve as an introduction to R's
unque \emph{syntax}, as well an illustration of how other visualisations
presented in this chapter can be reproduced.

\subsection{R's syntax}

\subsubsection{Objects}

\subsubsection{Functions and arguments}

Most operations that are performed on objects are done using
\emph{functions}. Understanding functions and their various
\emph{arguments} is key to manipulating and visualising data in R: the
more functions and arguments you know, the more you will be able to do.
Functions, in broad terms, are operations that change objects in R from
one thing to another. In mathematical language, they \emph{map} sets of
numbers onto each other. Arguments are the variables or parameters that
are fed into functions to alter their behavior. In terms of R's syntax,
arguments are separated by commas within the curved brackets that follow
from the function's name. A source of confusion with arguments can be
that in some cases they can be inserted directly, wheras in others R
needs to be told which argument is being referred to, as illustrated in
the code below:

\begin{Shaded}
\begin{Highlighting}[]
\KeywordTok{seq}\NormalTok{(}\DataTypeTok{from =} \DecValTok{0}\NormalTok{, }\DataTypeTok{to =} \DecValTok{2}\NormalTok{, }\DataTypeTok{by =} \FloatTok{0.5}\NormalTok{)}
\end{Highlighting}
\end{Shaded}

\begin{verbatim}
## [1] 0.0 0.5 1.0 1.5 2.0
\end{verbatim}

\begin{Shaded}
\begin{Highlighting}[]
\KeywordTok{seq}\NormalTok{(}\DecValTok{0}\NormalTok{, }\DecValTok{2}\NormalTok{, }\FloatTok{0.5}\NormalTok{)}
\end{Highlighting}
\end{Shaded}

\begin{verbatim}
## [1] 0.0 0.5 1.0 1.5 2.0
\end{verbatim}

\begin{Shaded}
\begin{Highlighting}[]

\KeywordTok{seq}\NormalTok{(}\DecValTok{0}\NormalTok{, }\DecValTok{2}\NormalTok{, }\DataTypeTok{length.out =} \DecValTok{6}\NormalTok{)}
\end{Highlighting}
\end{Shaded}

\begin{verbatim}
## [1] 0.0 0.4 0.8 1.2 1.6 2.0
\end{verbatim}

\begin{Shaded}
\begin{Highlighting}[]
\KeywordTok{seq}\NormalTok{(}\DecValTok{0}\NormalTok{, }\DecValTok{2}\NormalTok{, }\DecValTok{6}\NormalTok{)}
\end{Highlighting}
\end{Shaded}

\begin{verbatim}
## [1] 0
\end{verbatim}

Before learning about specific functions for spatial analysis and
visualisation, it is worth taking some time to think about what a
function is and how the arguments passed to it affect how it works. The
function \texttt{plot} is a good example, because it can take many
different input datasets and arguments and produces very different
results depending on the arguments it is given. Let's start with a basic
example:

\begin{Shaded}
\begin{Highlighting}[]
\NormalTok{x <-}\StringTok{ }\DecValTok{1}\NormalTok{:}\DecValTok{20}
\NormalTok{y <-}\StringTok{ }\DecValTok{20} \NormalTok{*}\StringTok{ }\NormalTok{x^}\DecValTok{2} \NormalTok{-}\StringTok{ }\NormalTok{x^}\DecValTok{3}
\KeywordTok{plot}\NormalTok{(x, y)}
\end{Highlighting}
\end{Shaded}

\begin{figure}[htbp]
\centering
\includegraphics{figure/unnamed-chunk-2.png}
\caption{unnamed-chunk-2}
\end{figure}

In the above code, the funtion \texttt{plot} was given two arguments,
\texttt{x} and \texttt{y} and its default settings are to interpret
these as values on a cartesian coordinate system to plot.

\subsection{Chapter overview}

\section{Map Production: Best Practice}

Good maps depend on sound analysis and data preparation and can have an
enormous impact on the understanding and communication of results. It
has never been easier to produce a map. The underlying data required are
available in unprecedented volumes and the technological capabilities of
transforming them into compelling maps and graphics are increasingly
sophisticated and straightforward to use. Data and software, however,
only offer the starting points of good spatial data visualisation since
they need to be refined and calibrated by the researchers seeking to
communicate their findings. In this section we will run through the
features of a good map. We will then seek to emulate them with R in
Section XX. It is worth noting that not all good maps and graphics
contain all the features below -- they should simply be seen as
suggestions rather than firm principles.

Effective map making is hard process -- as Krygier and Wood (XXX) put it
``there is a lot to see, think about, and do'' (p6). It often comes at
the end of a period of intense data analysis and perhaps when the
priority is to get a paper finished or results published and can
therefore be rushed as a result. The beauty of R (and other scripting
languages) is the ability to save code and simply re-run it with
different data. Colours, map adornments and other parameters can
therefore be quickly applied so it is well worth creating a template
script that adheres to best practice.

We have selected ggplot2 as our package of choice for the bulk of our
maps and spatial data visualisations because it has a number of these
elements at its core. The ``gg'' in its slightly odd name stands for
``Grammar of Graphics'', which is a set of rules developed by Leland
Wilkinson (2005) in a book of the same name. Grammar in the context of
graphics works in much the same way as it does in language- it provides
a structure. The structure is informed by both human perception and also
mathematics to ensure that the resulting visualisations are both
technically sound and comprehensible. Through creating ggplot2, Hadley
Wickham, implemented these rules as well as developing ways in which
plots can be built up in layers (see Wickham, 2010). This layering
component is especially useful in the context of spatial data since it
is conceptually the same as map layers in Geographical Information
Systems (GIS).

First load the libraries required for this section:

\begin{Shaded}
\begin{Highlighting}[]
\KeywordTok{library}\NormalTok{(rgdal)}
\end{Highlighting}
\end{Shaded}

\begin{verbatim}
## Loading required package: sp
## rgdal: version: 0.8-14, (SVN revision 496)
## Geospatial Data Abstraction Library extensions to R successfully loaded
## Loaded GDAL runtime: GDAL 1.9.0, released 2011/12/29
## Path to GDAL shared files: /usr/share/gdal/1.9
## Loaded PROJ.4 runtime: Rel. 4.7.1, 23 September 2009, [PJ_VERSION: 470]
## Path to PROJ.4 shared files: (autodetected)
\end{verbatim}

\begin{Shaded}
\begin{Highlighting}[]
\KeywordTok{library}\NormalTok{(ggplot2)}
\KeywordTok{library}\NormalTok{(gridExtra)}
\end{Highlighting}
\end{Shaded}

\begin{verbatim}
## Loading required package: grid
\end{verbatim}

You will also need create a folder and then set it as your working
directory. Below we assume the name is \texttt{Uname}, and the folder is
saved as \texttt{sdvwR} in the Desktop in Windows.

\begin{Shaded}
\begin{Highlighting}[]
\KeywordTok{setwd}\NormalTok{(}\StringTok{"c:/Users/Uname/Desktop/sdvwR"}\NormalTok{)}
\end{Highlighting}
\end{Shaded}

For this section we are going to use a map of the world to demonstrate
some of the cartographic principles discussed. A world map is available
from the Natural Earth website. The code below will download this and
save it to your working directory.

\begin{Shaded}
\begin{Highlighting}[]
\KeywordTok{download.file}\NormalTok{(}\DataTypeTok{url =} \StringTok{"http://www.naturalearthdata.com/http//www.naturalearthdata.com/download/110m/cultural/ne_110m_admin_0_countries.zip"}\NormalTok{, }
    \StringTok{"ne_110m_admin_0_countries.zip"}\NormalTok{, }\StringTok{"auto"}\NormalTok{)}
\KeywordTok{unzip}\NormalTok{(}\StringTok{"ne_110m_admin_0_countries.zip"}\NormalTok{, }\DataTypeTok{exdir =} \StringTok{"data/"}\NormalTok{)  }\CommentTok{# unzip to data folder}
\KeywordTok{file.remove}\NormalTok{(}\StringTok{"ne_110m_admin_0_countries.zip"}\NormalTok{)  }\CommentTok{# remove zip file}
\end{Highlighting}
\end{Shaded}

\begin{verbatim}
## [1] TRUE
\end{verbatim}

Once downloaded we can then load the data into the R console. We have
just downloaded a shapefile, which as Section XX explains, is not
handled as a ``standard'' data object in R.

\begin{Shaded}
\begin{Highlighting}[]
\NormalTok{wrld <-}\StringTok{ }\KeywordTok{readOGR}\NormalTok{(}\StringTok{"data/"}\NormalTok{, }\StringTok{"ne_110m_admin_0_countries"}\NormalTok{)}
\end{Highlighting}
\end{Shaded}

\begin{verbatim}
## OGR data source with driver: ESRI Shapefile 
## Source: "data/", layer: "ne_110m_admin_0_countries"
## with 177 features and 63 fields
## Feature type: wkbPolygon with 2 dimensions
\end{verbatim}

\begin{Shaded}
\begin{Highlighting}[]
\KeywordTok{plot}\NormalTok{(wrld)}
\end{Highlighting}
\end{Shaded}

\begin{figure}[htbp]
\centering
\includegraphics{figure/Initial_plot_of_world_boundaries.png}
\caption{Initial plot of world boundaries}
\end{figure}

To see the first ten rows of attribute information assocuiated with each
of the country boundaries type the following

\begin{Shaded}
\begin{Highlighting}[]
\KeywordTok{head}\NormalTok{(wrld@data)}
\end{Highlighting}
\end{Shaded}

You can see there are a lot of columns associated with this file.
Although we will keep all of the them, we are only really interested in
the population estimate (``pop\_est'') field. Before progressing it is
is worth reprojecting the data in order that the population data can be
seen better. The coordinate reference system of the wrld shapefile is
currently WGS84. This the common latitude and longitude format that all
spatial software packages understand. From a cartographic perspective
the standard plots of this projection, of the kind produced above, are
not suitable since they distort the shapes of those countries further
from the equator. Instead the Robinson projection provides a good
compromise between areal distortion and shape preservation. We therefore
project it as follows.

\begin{Shaded}
\begin{Highlighting}[]
\KeywordTok{library}\NormalTok{(geosphere)}
\NormalTok{wrld.rob <-}\StringTok{ }\KeywordTok{spTransform}\NormalTok{(wrld, }\KeywordTok{CRS}\NormalTok{(}\StringTok{"+proj=robin"}\NormalTok{))}
\KeywordTok{plot}\NormalTok{(wrld.rob)}
\end{Highlighting}
\end{Shaded}

\begin{figure}[htbp]
\centering
\includegraphics{figure/unnamed-chunk-5.png}
\caption{unnamed-chunk-5}
\end{figure}

``ESRI: 54030'' is the reference code of the Robinson prjection in the
database of projections that R downloads with the rgdal package. You
will have spotted from the plot that the countries in the world map are
much better proportioned.

We now need to ``fortify'' this spatial data to convert it into a format
that ggplot2 understands, we also use ``merge'' to re-attach the
attribute data that is lost in the fortify operation.

\begin{Shaded}
\begin{Highlighting}[]
\CommentTok{# fortify requires rgeos or maptools packages - have we already loaded it?}
\CommentTok{# !!!}
\NormalTok{wrld.rob.f <-}\StringTok{ }\KeywordTok{fortify}\NormalTok{(wrld.rob, }\DataTypeTok{region =} \StringTok{"sov_a3"}\NormalTok{)}
\end{Highlighting}
\end{Shaded}

\begin{verbatim}
## Loading required package: rgeos
## rgeos version: 0.3-2, (SVN revision 413M)
##  GEOS runtime version: 3.3.3-CAPI-1.7.4 
##  Polygon checking: TRUE
\end{verbatim}

\begin{Shaded}
\begin{Highlighting}[]

\NormalTok{wrld.pop.f <-}\StringTok{ }\KeywordTok{merge}\NormalTok{(wrld.rob.f, wrld.rob@data, }\DataTypeTok{by.x =} \StringTok{"id"}\NormalTok{, }\DataTypeTok{by.y =} \StringTok{"sov_a3"}\NormalTok{)}
\end{Highlighting}
\end{Shaded}

\begin{Shaded}
\begin{Highlighting}[]
\CommentTok{# continuous colour ramp}

\NormalTok{map <-}\StringTok{ }\KeywordTok{ggplot}\NormalTok{(wrld.pop.f, }\KeywordTok{aes}\NormalTok{(long, lat, }\DataTypeTok{group =} \NormalTok{group, }\DataTypeTok{fill =} \NormalTok{pop_est)) +}\StringTok{ }\KeywordTok{geom_polygon}\NormalTok{() +}\StringTok{ }
\StringTok{    }\KeywordTok{coord_equal}\NormalTok{() +}\StringTok{ }\KeywordTok{labs}\NormalTok{(}\DataTypeTok{x =} \StringTok{"Longitude"}\NormalTok{, }\DataTypeTok{y =} \StringTok{"Latitude"}\NormalTok{, }\DataTypeTok{fill =} \StringTok{"World Population"}\NormalTok{) +}\StringTok{ }
\StringTok{    }\KeywordTok{ggtitle}\NormalTok{(}\StringTok{"World Population"}\NormalTok{)}

\CommentTok{# better colours with more breaks- to finish}

\NormalTok{map +}\StringTok{ }\KeywordTok{scale_fill_continuous}\NormalTok{(}\DataTypeTok{breaks =} \KeywordTok{c}\NormalTok{(}\DecValTok{10}\NormalTok{^}\KeywordTok{c}\NormalTok{(}\DecValTok{8}\NormalTok{, }\DecValTok{9}\NormalTok{)))}
\end{Highlighting}
\end{Shaded}

\begin{figure}[htbp]
\centering
\includegraphics{figure/unnamed-chunk-7.png}
\caption{unnamed-chunk-7}
\end{figure}

\begin{Shaded}
\begin{Highlighting}[]

\CommentTok{# categorical variables}
\end{Highlighting}
\end{Shaded}

\section{Conforming to colour conventions}

Colour has an enormous impact on how people will percieve your graphic.
``Readers'' of a map come to it with a range of pre-conceptions about
how the world looks. If the map's purpose is to clearly communicate data
then it is often advisable to conform to conventions so as not to
disorientate readers to ensure they can focus on the key messages
contained in the data. A good example of this is the use of blue for
bodies of water and green for landmass. The code example below generates
two plots with our wrld.pop.f object. The first colours the land blue
and the sea (in this case the background to the map) green and the
second is more conventional. We use the ``grid.arrange'' function from
the ``gridExtra'' package to display the maps side by side.

\begin{Shaded}
\begin{Highlighting}[]
\NormalTok{map2 <-}\StringTok{ }\KeywordTok{ggplot}\NormalTok{(wrld.pop.f, }\KeywordTok{aes}\NormalTok{(long, lat, }\DataTypeTok{group =} \NormalTok{group)) +}\StringTok{ }\KeywordTok{coord_equal}\NormalTok{()}

\NormalTok{blue <-}\StringTok{ }\NormalTok{map2 +}\StringTok{ }\KeywordTok{geom_polygon}\NormalTok{(}\DataTypeTok{fill =} \StringTok{"light blue"}\NormalTok{) +}\StringTok{ }\KeywordTok{theme}\NormalTok{(}\DataTypeTok{panel.background =} \KeywordTok{element_rect}\NormalTok{(}\DataTypeTok{fill =} \StringTok{"dark green"}\NormalTok{))}

\NormalTok{green <-}\StringTok{ }\NormalTok{map2 +}\StringTok{ }\KeywordTok{geom_polygon}\NormalTok{(}\DataTypeTok{fill =} \StringTok{"dark green"}\NormalTok{) +}\StringTok{ }\KeywordTok{theme}\NormalTok{(}\DataTypeTok{panel.background =} \KeywordTok{element_rect}\NormalTok{(}\DataTypeTok{fill =} \StringTok{"light blue"}\NormalTok{))}

\KeywordTok{grid.arrange}\NormalTok{(blue, green, }\DataTypeTok{ncol =} \DecValTok{2}\NormalTok{)}
\end{Highlighting}
\end{Shaded}

\begin{figure}[htbp]
\centering
\includegraphics{figure/unnamed-chunk-8.png}
\caption{unnamed-chunk-8}
\end{figure}

\section{Experimenting with line colour and line widths}

In addition to conforming to colour conventions, line colour and width
offer important parameters, which are often overlooked tools for
increasing the legibility of a graphic. As the code below demonstrates,
it is possible to adjust line colour through using the ``colour''
parameter and the line width using the ``lwd'' parameter. The impact of
different line widths will vary depending on your screen size and
resolution. If you save the plot to pdf (or an image) then the size at
which you do this will also affect the line widths.

\begin{Shaded}
\begin{Highlighting}[]
\NormalTok{map3 <-}\StringTok{ }\NormalTok{map2 +}\StringTok{ }\KeywordTok{theme}\NormalTok{(}\DataTypeTok{panel.background =} \KeywordTok{element_rect}\NormalTok{(}\DataTypeTok{fill =} \StringTok{"light blue"}\NormalTok{))}

\NormalTok{yellow <-}\StringTok{ }\NormalTok{map3 +}\StringTok{ }\KeywordTok{geom_polygon}\NormalTok{(}\DataTypeTok{fill =} \StringTok{"dark green"}\NormalTok{, }\DataTypeTok{colour =} \StringTok{"yellow"}\NormalTok{)}

\NormalTok{black <-}\StringTok{ }\NormalTok{map3 +}\StringTok{ }\KeywordTok{geom_polygon}\NormalTok{(}\DataTypeTok{fill =} \StringTok{"dark green"}\NormalTok{, }\DataTypeTok{colour =} \StringTok{"black"}\NormalTok{)}

\NormalTok{thin <-}\StringTok{ }\NormalTok{map3 +}\StringTok{ }\KeywordTok{geom_polygon}\NormalTok{(}\DataTypeTok{fill =} \StringTok{"dark green"}\NormalTok{, }\DataTypeTok{colour =} \StringTok{"black"}\NormalTok{, }\DataTypeTok{lwd =} \FloatTok{0.1}\NormalTok{)}

\NormalTok{thick <-}\StringTok{ }\NormalTok{map3 +}\StringTok{ }\KeywordTok{geom_polygon}\NormalTok{(}\DataTypeTok{fill =} \StringTok{"dark green"}\NormalTok{, }\DataTypeTok{colour =} \StringTok{"black"}\NormalTok{, }\DataTypeTok{lwd =} \FloatTok{1.5}\NormalTok{)}

\KeywordTok{grid.arrange}\NormalTok{(yellow, black, thick, thin, }\DataTypeTok{ncol =} \DecValTok{2}\NormalTok{)}
\end{Highlighting}
\end{Shaded}

\begin{figure}[htbp]
\centering
\includegraphics{figure/unnamed-chunk-9.png}
\caption{unnamed-chunk-9}
\end{figure}

There are other parameters such as layer transparency that can be
applied to all aspects of the plot - both points, lines and polygons -
that we will reference in later examples in this chapter.

\section{Map Adornments and Annotations}

Map adornments and annotations are essential to orientate the viewer and
provide context; they include graticules, north arrows, scale bars and
data attribution. Not all are required on a single map, indeed it is
often best that they are used sparingly to avoid unecessary clutter
(Monkhouse and Wilkinson, 1971). Unfortunately it is not always as
straightforward to add these in R, and perhaps less so using the ggplot2
paradigm, when compared to a conventional GIS. Here we will outline the
ways in which annotations can be added.

!!!! In the maps created so far, we have defined the \emph{aesthetics}
of the map in the foundation function \texttt{ggplot}. The result of
this is that all subsequent layers are expected to have the same
variables and essentially contain data with the same dimensions as
original dataset. But what if we want to add a new layer from a
completely different dataset To do this, we must not add any arguments
to the \texttt{ggplot} function, only adding data sources one layer at a
time:

\section{North arrow}

\begin{Shaded}
\begin{Highlighting}[]
\KeywordTok{ggplot}\NormalTok{() +}\StringTok{ }\KeywordTok{geom_polygon}\NormalTok{(}\DataTypeTok{data =} \NormalTok{wrld.pop.f, }\KeywordTok{aes}\NormalTok{(long, lat, }\DataTypeTok{group =} \NormalTok{group, }\DataTypeTok{fill =} \NormalTok{pop_est)) +}\StringTok{ }
\StringTok{    }\KeywordTok{geom_line}\NormalTok{(}\KeywordTok{aes}\NormalTok{(}\DataTypeTok{x =} \KeywordTok{c}\NormalTok{(-}\DecValTok{160}\NormalTok{, -}\DecValTok{160}\NormalTok{), }\DataTypeTok{y =} \KeywordTok{c}\NormalTok{(}\DecValTok{0}\NormalTok{, }\DecValTok{25}\NormalTok{)), }\DataTypeTok{arrow =} \KeywordTok{arrow}\NormalTok{())}
\end{Highlighting}
\end{Shaded}

\begin{figure}[htbp]
\centering
\includegraphics{figure/unnamed-chunk-10.png}
\caption{unnamed-chunk-10}
\end{figure}

\begin{Shaded}
\begin{Highlighting}[]


\CommentTok{# scale bar- found this function}

\NormalTok{hscale_segment =}\StringTok{ }\NormalTok{function(breaks, ...) \{}
    \NormalTok{y =}\StringTok{ }\KeywordTok{unique}\NormalTok{(breaks$y)}
    \KeywordTok{stopifnot}\NormalTok{(}\KeywordTok{length}\NormalTok{(y) ==}\StringTok{ }\DecValTok{1}\NormalTok{)}
    \NormalTok{dx =}\StringTok{ }\KeywordTok{max}\NormalTok{(breaks$x) -}\StringTok{ }\KeywordTok{min}\NormalTok{(breaks$x)}
    \NormalTok{dy =}\StringTok{ }\DecValTok{1}\NormalTok{/}\DecValTok{30} \NormalTok{*}\StringTok{ }\NormalTok{dx}
    \NormalTok{hscale =}\StringTok{ }\KeywordTok{data.frame}\NormalTok{(}\DataTypeTok{ix =} \KeywordTok{min}\NormalTok{(breaks$x), }\DataTypeTok{iy =} \NormalTok{y, }\DataTypeTok{jx =} \KeywordTok{max}\NormalTok{(breaks$x), }\DataTypeTok{jy =} \NormalTok{y)}
    \NormalTok{vticks =}\StringTok{ }\KeywordTok{data.frame}\NormalTok{(}\DataTypeTok{ix =} \NormalTok{breaks$x, }\DataTypeTok{iy =} \NormalTok{(y -}\StringTok{ }\NormalTok{dy), }\DataTypeTok{jx =} \NormalTok{breaks$x, }\DataTypeTok{jy =} \NormalTok{(y +}\StringTok{ }
\StringTok{        }\NormalTok{dy))}
    \NormalTok{df =}\StringTok{ }\KeywordTok{rbind}\NormalTok{(hscale, vticks)}
    \KeywordTok{return}\NormalTok{(}\KeywordTok{geom_segment}\NormalTok{(}\DataTypeTok{data =} \NormalTok{df, }\KeywordTok{aes}\NormalTok{(}\DataTypeTok{x =} \NormalTok{ix, }\DataTypeTok{xend =} \NormalTok{jx, }\DataTypeTok{y =} \NormalTok{iy, }\DataTypeTok{yend =} \NormalTok{jy), }
        \NormalTok{...))}
    
\NormalTok{\}}

\NormalTok{hscale_text =}\StringTok{ }\NormalTok{function(breaks, ...) \{}
    \NormalTok{dx =}\StringTok{ }\KeywordTok{max}\NormalTok{(breaks$x) -}\StringTok{ }\KeywordTok{min}\NormalTok{(breaks$x)}
    \NormalTok{dy =}\StringTok{ }\DecValTok{2}\NormalTok{/}\DecValTok{30} \NormalTok{*}\StringTok{ }\NormalTok{dx}
    \NormalTok{breaks$y =}\StringTok{ }\NormalTok{breaks$y +}\StringTok{ }\NormalTok{dy}
    \KeywordTok{return}\NormalTok{(}\KeywordTok{geom_text}\NormalTok{(}\DataTypeTok{data =} \NormalTok{breaks, }\KeywordTok{aes}\NormalTok{(}\DataTypeTok{x =} \NormalTok{x, }\DataTypeTok{y =} \NormalTok{y, }\DataTypeTok{label =} \NormalTok{label), }\DataTypeTok{hjust =} \FloatTok{0.5}\NormalTok{, }
        \DataTypeTok{vjust =} \DecValTok{0}\NormalTok{, ...))}
    
\NormalTok{\}}
\end{Highlighting}
\end{Shaded}

There is an almost infinite number of different combinations of the
above parameters so take inspiration from maps and graphics you have
seen and liked. The process is an iterative one, it will take multiple
attempts to get right. Show your map to friends and colleagues- all will
have an opinion but don't be afraid to stand by the decisions you have
taken.

Consistency- across papers.

\section{R and Spatial Data}

\subsection{Spatial Data in R}

In any data analysis project, spatial or otherwise, it is important to
have a strong understanding of the dataset before progressing. This
section will therefore begin with a description of the input data used
in this section. We will see how data can be loaded into R and exported
to other formats, before going into more detail about the underlying
structure of spatial data in R: how it `sees' spatial data is quite
unique.

\subsubsection{Loading spatial data in R}

In most situations, the starting point of spatial analysis tasks is
loading in pre-existing datasets. These may originate from government
agencies, remote sensing devices or `volunteered geographical
information' from GPS devices, online databases such as Open Street Map
or geo-tagged social media (Goodchild 2007). The diversity of
geographical data formats is large.

R is able to import a very wide range of spatial data formats thanks to
its interface with the Geospatial Data Abstraction Library (GDAL), which
is enabled by loading the package \texttt{rgdal} into R. Below we will
load data from two spatial data formats: GPS eXchange (\texttt{.gpx})
and an ESRI Shapefile (consisting of at least files with \texttt{.shp},
\texttt{.shx} and \texttt{.dbf} extensions).

\texttt{readOGR} is in fact cabable of loading dozens more file formats,
so the focus is on the \emph{method} rather than the specific formats.
The `take home message' is that the \texttt{readOGR} function is capable
of loading most common spatial file formats, but behaves differently
depending on file type. Let's start with a \texttt{.gpx} file, a
tracklog recording a bicycle ride from Sheffield to Wakefield which was
uploaded Open Street Map. {[}!!! more detail?{]}

\begin{Shaded}
\begin{Highlighting}[]
\CommentTok{# download.file('http://www.openstreetmap.org/trace/1619756/data', destfile}
\CommentTok{# = 'data/gps-trace.gpx')}
\KeywordTok{library}\NormalTok{(rgdal)  }\CommentTok{# load the gdal package}
\end{Highlighting}
\end{Shaded}

\begin{verbatim}
## Loading required package: sp
## rgdal: version: 0.8-11, (SVN revision 479M)
## Geospatial Data Abstraction Library extensions to R successfully loaded
## Loaded GDAL runtime: GDAL 1.9.2, released 2012/10/08
## Path to GDAL shared files: /usr/share/gdal
## Loaded PROJ.4 runtime: Rel. 4.8.0, 6 March 2012, [PJ_VERSION: 480]
## Path to PROJ.4 shared files: (autodetected)
\end{verbatim}

\begin{Shaded}
\begin{Highlighting}[]
\KeywordTok{ogrListLayers}\NormalTok{(}\DataTypeTok{dsn =} \StringTok{"data/gps-trace.gpx"}\NormalTok{)  }\CommentTok{# which layers are available?}
\end{Highlighting}
\end{Shaded}

\begin{verbatim}
## [1] "waypoints"    "routes"       "tracks"       "route_points"
## [5] "track_points"
\end{verbatim}

\begin{Shaded}
\begin{Highlighting}[]
\NormalTok{shf2lds <-}\StringTok{ }\KeywordTok{readOGR}\NormalTok{(}\DataTypeTok{dsn =} \StringTok{"data/gps-trace.gpx"}\NormalTok{, }\DataTypeTok{layer =} \StringTok{"tracks"}\NormalTok{)  }\CommentTok{# load track}
\end{Highlighting}
\end{Shaded}

\begin{verbatim}
## OGR data source with driver: GPX 
## Source: "data/gps-trace.gpx", layer: "tracks"
## with 1 features and 12 fields
## Feature type: wkbMultiLineString with 2 dimensions
\end{verbatim}

\begin{Shaded}
\begin{Highlighting}[]
\KeywordTok{plot}\NormalTok{(shf2lds)}
\NormalTok{shf2lds.p <-}\StringTok{ }\KeywordTok{readOGR}\NormalTok{(}\DataTypeTok{dsn =} \StringTok{"data/gps-trace.gpx"}\NormalTok{, }\DataTypeTok{layer =} \StringTok{"track_points"}\NormalTok{)  }\CommentTok{# load points}
\end{Highlighting}
\end{Shaded}

\begin{verbatim}
## OGR data source with driver: GPX 
## Source: "data/gps-trace.gpx", layer: "track_points"
## with 6085 features and 26 fields
## Feature type: wkbPoint with 2 dimensions
\end{verbatim}

\begin{Shaded}
\begin{Highlighting}[]
\KeywordTok{points}\NormalTok{(shf2lds.p[}\KeywordTok{seq}\NormalTok{(}\DecValTok{1}\NormalTok{, }\DecValTok{3000}\NormalTok{, }\DecValTok{100}\NormalTok{), ])}
\end{Highlighting}
\end{Shaded}

\begin{figure}[htbp]
\centering
\includegraphics{figure/Leeds_to_Sheffield_GPS_data.png}
\caption{Leeds to Sheffield GPS data}
\end{figure}

There is a lot going on in the preceding 7 lines of code, including
functions that you are unlikely to have encountered before. Let us think
about what has happened, line-by-line.

First, we used R to \emph{download} a file from the internet, using the
function \texttt{download.file}. The two essential arguments of this
function are \texttt{url} (we could have typed\texttt{url =} before the
link) and \texttt{destfile} (which means destination file). As with any
function, more optional arguments can be viewed by typing
\texttt{?download.file}.

When \texttt{rgdal} has succesfully loaded, the next task is not to
import the file directly, but to find out which \emph{layers} are
available to import, with the function \texttt{ogrListLayers}. The
output from this command tells us that various layers are available,
including \texttt{tracks} and \texttt{track\_points}, which we
subsequently load using \texttt{readOGR}. The basic \texttt{plot}
function is used to plot the newly imported objects, ensuring they make
sense. In the second \texttt{plot} function, we take a subset of the
object (see section \ldots{} for more on this).

As stated in the help documentation (accessed by entering
\texttt{?readOGR}), the \texttt{dsn =} argument is interpreted
differently depending on the type of file used. In the above example,
the filename was the data source name. To load Shapefiles, by contrast,
the \emph{folder} containing the data is used:

\begin{Shaded}
\begin{Highlighting}[]
\NormalTok{lnd <-}\StringTok{ }\KeywordTok{readOGR}\NormalTok{(}\DataTypeTok{dsn =} \StringTok{"data/"}\NormalTok{, }\StringTok{"london_sport"}\NormalTok{)}
\end{Highlighting}
\end{Shaded}

Here, the data is assumed to reside in a folder entitled \texttt{data}
which in R's current working directory (remember to check this using
\texttt{getwd()}). If the files were stored in the working directory,
one would use \texttt{dsn = "."} instead. Again, it may be wise to plot
the data that results, to ensure that it has worked correctly. Now that
the data has been loaded into R's own \texttt{sp} format, try
interogating and plotting it, using functions such as \texttt{summary}
and \texttt{plot}.

\subsubsection{The size of spatial datasets in R}

Any data that has been read into R's \emph{workspace}, which constitutes
all objects that can be accessed by name and can be listed using the
\texttt{ls()} function, can be saved in R's own data storage file type,
\texttt{.RData}. Spatial datasets can get quite large and this can cause
problems on computers by consuming all available random access memory
(RAM) or hard disk space available to the computer. It is therefore wise
to understand roughly how large spatial objects are; this will also
provide insight into how long certain functions will take to run.

In the absence of prior knowledge, which of the two objects loaded in
the previous section would be expected to take up more memory. One could
hypothesise that the London boroughs represented by the object
\texttt{lnd} would be larger, but how much larger? We could simply look
at the size of the associated files, but R also provides a function
(\texttt{object.size}) for discovering how large objects loaded into its
workspace are:

\begin{Shaded}
\begin{Highlighting}[]
\KeywordTok{object.size}\NormalTok{(shf2lds)}
\end{Highlighting}
\end{Shaded}

\begin{verbatim}
## 107464 bytes
\end{verbatim}

\begin{Shaded}
\begin{Highlighting}[]
\KeywordTok{object.size}\NormalTok{(lnd)}
\end{Highlighting}
\end{Shaded}

\begin{verbatim}
## 125544 bytes
\end{verbatim}

Surprisingly, the GPS data is larger. To see why, we can find out how
many \emph{vertices} (points connected by lines) are contained in each
dataset:

\begin{Shaded}
\begin{Highlighting}[]
\KeywordTok{sapply}\NormalTok{(lnd@polygons, function(x) }\KeywordTok{length}\NormalTok{(x))}
\end{Highlighting}
\end{Shaded}

\begin{verbatim}
##  [1] 1 1 1 1 1 1 1 1 1 1 1 1 1 1 1 1 1 1 1 1 1 1 1 1 1 1 1 1 1 1 1 1 1
\end{verbatim}

\begin{Shaded}
\begin{Highlighting}[]
\NormalTok{x <-}\StringTok{ }\KeywordTok{sapply}\NormalTok{(lnd@polygons, function(x) }\KeywordTok{nrow}\NormalTok{(x@Polygons[[}\DecValTok{1}\NormalTok{]]@coords))}
\KeywordTok{sum}\NormalTok{(x)}
\end{Highlighting}
\end{Shaded}

\begin{verbatim}
## [1] 1102
\end{verbatim}

\begin{Shaded}
\begin{Highlighting}[]

\KeywordTok{sapply}\NormalTok{(shf2lds@lines, function(x) }\KeywordTok{length}\NormalTok{(x))}
\end{Highlighting}
\end{Shaded}

\begin{verbatim}
## [1] 1
\end{verbatim}

\begin{Shaded}
\begin{Highlighting}[]
\KeywordTok{sapply}\NormalTok{(shf2lds@lines, function(x) }\KeywordTok{nrow}\NormalTok{(x@Lines[[}\DecValTok{1}\NormalTok{]]@coords))}
\end{Highlighting}
\end{Shaded}

\begin{verbatim}
## [1] 6085
\end{verbatim}

It is quite likely that the above code little sense at first; the
important thing to remember is that for each object we performed two
functions: 1) a check that each line or polygon consists only of a
single \emph{part} (that can be joined to attribut data) and 2) the use
of \texttt{nrow} to count the number of vertices. The use of the
\texttt{@} symbol should seem strange - its meaning will become clear in
the section !!!. (Note also that the function \texttt{fortify},
discussed in section !!!, can also be used to extract the vertice count
of spatial objects in R.)

Without worrying, for now, about how these vertice counts were
performed, it is clear that the GPS data has almost 6 times the number
of vertices as does the London data, explaining its larger size. Yet
when plotted, the GPS data does not seem more detailed, implying that
some of the vertices in the object are not needed for visualisation at
the scale of the objects \emph{bounding box}.

\subsubsection{Simplifying geometries}

The wastefulness of the GPS data for visualisation (the full dataset may
be useful for other types of analysis) raises the question following
question: can the object be simplified such that its key features
features remain while substantially reducing its size? The answer is
yes. In the code below, we harness the power of the \texttt{rgeos}
package and its \texttt{gSimplify} function to simplify spatial R
objects (the code can also be used to simplify polygon geometries):

\begin{Shaded}
\begin{Highlighting}[]
\KeywordTok{library}\NormalTok{(rgeos)}
\end{Highlighting}
\end{Shaded}

\begin{verbatim}
## rgeos version: 0.3-2, (SVN revision 413M)
##  GEOS runtime version: 3.3.9-CAPI-1.7.9 
##  Polygon checking: TRUE
\end{verbatim}

\begin{Shaded}
\begin{Highlighting}[]
\NormalTok{shf2lds.simple <-}\StringTok{ }\KeywordTok{gSimplify}\NormalTok{(shf2lds, }\DataTypeTok{tol =} \FloatTok{0.001}\NormalTok{)}
\NormalTok{(}\KeywordTok{object.size}\NormalTok{(shf2lds.simple)/}\KeywordTok{object.size}\NormalTok{(shf2lds))[}\DecValTok{1}\NormalTok{]}
\end{Highlighting}
\end{Shaded}

\begin{verbatim}
## [1] 0.04608
\end{verbatim}

\begin{Shaded}
\begin{Highlighting}[]
\KeywordTok{plot}\NormalTok{(shf2lds.simple)}
\KeywordTok{plot}\NormalTok{(shf2lds, }\DataTypeTok{col =} \StringTok{"red"}\NormalTok{, }\DataTypeTok{add =} \NormalTok{T)}
\end{Highlighting}
\end{Shaded}

In the above block of code, \texttt{gSimplify} is given the object
\texttt{shf2lds} and the \texttt{tol} argument, short for ``tolerance'',
is set at 0.001 (much larger values may be needed, for data that use is
\emph{projected} - does not use latitude and longitude). Comparison
between the sizes of the simplified object and the orginal shows that
the new object is less than 3\% of its original size. Try plotting the
orginal and simplified tracks on your computer: when visualised using
the \texttt{plot} function, it becomes clear that the object
\texttt{shf2lds.simple} retains the overall shape of the line and is
virtually indistinguishable from the orginal object.

This example is rather contrived because even the larger object
\texttt{shf2lds} is only 0.107 Mb, negligible compared with the
gigabytes of RAM available to modern computers. However, it underlines a
wider point: for \emph{visualisation} purposes at small spatial scales
(i.e.~covering a large area of the Earth on a small map), the
\emph{geometries} associated with spatial data can often be simplified
to reduce processing time and usage of RAM. The other advantage of
simplification is that it reduces the size occupied by spatial datasets
when they are saved.

\subsubsection{Saving and exporting spatial objects}

\subsection{The structure of spatial data in R}

\subsubsection{Spatial* data}

\paragraph{Points}

\paragraph{Lines}

\paragraph{Polygons}

\paragraph{Grids and raster data}

\subsubsection{`Flattening' data with \texttt{fortify}}

\subsection{The main spatial packages}

\subsubsection{sp}

\subsubsection{rgdal}

\subsubsection{rgeos}

\subsection{Maps with ggplot2}

\subsubsection{Adding base maps with ggmap}

\subsection{Manipulating spatial data}

\subsubsection{Coordinate reference systems and transformations}

\subsubsection{Attribute joins}

\subsubsection{Spatial joins}

A spatial join, like attribute joins, is used to transfer information
from one dataset to another. There is a clearly defined direction to
spatial joins, with the \emph{target layer} receiving information from
another spatial layer based on the proximity of elements from both
layers to each other. There are three broad types of spatial join:
one-to-one, many-to-one and one-to-many. We will focus only the former
two as the third type is rarely used.

One-to-one spatial joins are by far the easiest to understand and
compute because they simply involve the transfer of attributes in one
layer to another, based on location. A one-to-one join is depicted in
figure x below.

\begin{figure}[htbp]
\centering
\includegraphics{figure/Illustration_of_a_one-to-one_spatial_join_.png}
\caption{Illustration of a one-to-one spatial join}
\end{figure}

Many-to-one spatial joins involve taking a spatial layer with many
elements and allocating the attributes associated with these elements to
relatively few elements in the target spatial layer. A common type of
many-to-one spatial join is the allocation of data collected at many
point sources unevenly scattered over space to polygons representing
administrative boundaries, as represented in Fig. x.

\begin{Shaded}
\begin{Highlighting}[]
\NormalTok{lnd.stations <-}\StringTok{ }\KeywordTok{readOGR}\NormalTok{(}\StringTok{"data/"}\NormalTok{, }\StringTok{"lnd-stns"}\NormalTok{, }\DataTypeTok{p4s =} \StringTok{"+init=epsg:27700"}\NormalTok{)}
\end{Highlighting}
\end{Shaded}

\begin{verbatim}
## OGR data source with driver: ESRI Shapefile 
## Source: "data/", layer: "lnd-stns"
## with 2532 features and 6 fields
## Feature type: wkbPoint with 2 dimensions
\end{verbatim}

\begin{Shaded}
\begin{Highlighting}[]
\KeywordTok{plot}\NormalTok{(lnd)}
\KeywordTok{plot}\NormalTok{(lnd.stations[}\KeywordTok{round}\NormalTok{(}\KeywordTok{runif}\NormalTok{(}\DataTypeTok{n =} \DecValTok{500}\NormalTok{, }\DataTypeTok{min =} \DecValTok{1}\NormalTok{, }\DataTypeTok{max =} \KeywordTok{nrow}\NormalTok{(lnd.stations))), }
    \NormalTok{], }\DataTypeTok{add =} \NormalTok{T)}
\end{Highlighting}
\end{Shaded}

\begin{figure}[htbp]
\centering
\includegraphics{figure/Input_data_for_a_spatial_join.png}
\caption{Input data for a spatial join}
\end{figure}

The above code reads in a \texttt{SpatialPointsDataFrame} consisting of
2532 transport nodes in and surrounding London and then plots a random
sample of 500 of these over the previously loaded borough level
adminsitrative boundaries. The reason for ploting a sample of the points
rather than all of them is that the boundary data becomes difficult to
see if all of the points are ploted. It is also useful to see and
practice sampling techniques in practice; try to plot only the first 500
points, rather than a random selection, and describe the difference.

The most obvious issue with the point data from the perspective of a
spatial join with the borough data is that many of the points in the
dataset are in fact located outside the region of interest. Thus, the
first stage in the analysis is to filter the point data such that only
those that lie within London's administrative zones are selected. This
in itself is a kind of spatial join, and can be accomplished with the
following code.

\begin{Shaded}
\begin{Highlighting}[]
\KeywordTok{proj4string}\NormalTok{(lnd) <-}\StringTok{ }\KeywordTok{proj4string}\NormalTok{(lnd.stations)}
\end{Highlighting}
\end{Shaded}

\begin{verbatim}
## Warning: A new CRS was assigned to an object with an existing CRS:
## +proj=tmerc +lat_0=49 +lon_0=-2 +k=0.9996012717 +x_0=400000 +y_0=-100000 +ellps=airy +units=m +no_defs
## without reprojecting.
## For reprojection, use function spTransform in package rgdal
\end{verbatim}

\begin{Shaded}
\begin{Highlighting}[]
\NormalTok{lnd.stations <-}\StringTok{ }\NormalTok{lnd.stations[lnd, ]  }\CommentTok{# select only points within lnd}
\KeywordTok{plot}\NormalTok{(lnd.stations)  }\CommentTok{# check the result}
\end{Highlighting}
\end{Shaded}

\begin{figure}[htbp]
\centering
\includegraphics{figure/A_spatial_subset_of_the_points.png}
\caption{A spatial subset of the points}
\end{figure}

The station points now clearly follow the form of the \texttt{lnd}
shape, indicating that the procedure worked. Let's review the code that
allowed this to happen: the first line ensured that the CRS associated
with each layer is \emph{exactly} the same: this step should not be
required in most cases, but it is worth knowing about. Of course, if the
coordinate systems are \emph{actually} different in each layer, the
function \texttt{spTransform} will be needed to make them compatible.
This procedure is discussed in section !!!. In this case, only the name
was slightly different hence direct alteration of the CRS name via the
function \texttt{proj4string}.

The second line of code is where the magic happens and the brilliance of
R's sp package becomes clear: all that was needed was to place another
spatial object in the row index of the points (\texttt{{[}lnd, {]}}) and
R automatically understood that a subset based on location should be
produced. This line of code is an example of R's `terseness' - only a
single line of code is needed to perform what is in fact quite a complex
operation.

\subsection{Spatial aggregation}

Now that only stations which \emph{intersect} with the \texttt{lnd}
polygon have been selected, the next stage is to extract information
about the points within each zone. This many-to-one spatial join is also
known as \emph{spatial aggregation}. To do this there are a couple of
approaches: one using the \texttt{sp} package and the other using
\texttt{rgeos} (see Bivand et al. 2013, 5.3).

As with the \emph{spatial subest} method described above, the developers
of R have been very clever in their implementation of spatial
aggregations methods. To minimise typing and ensure consistency with R's
base functions, \texttt{sp} extends the capabilities of the
\texttt{aggregate} function to automatically detect whether the user is
asking for a spatial or a non-spatial aggregation (they are, in essence,
the same thing - we recommend learning about the non-spatial use of
\texttt{aggregate} in R for comparison).

Continuing with the example of station points in London polygons, let us
use the spatial extension of \texttt{aggregate} to count how many points
are in each borough:

\begin{Shaded}
\begin{Highlighting}[]
\NormalTok{lndStC <-}\StringTok{ }\KeywordTok{aggregate}\NormalTok{(lnd.stations, }\DataTypeTok{by =} \NormalTok{lnd, }\DataTypeTok{FUN =} \NormalTok{length)}
\KeywordTok{summary}\NormalTok{(lndStC)}
\KeywordTok{plot}\NormalTok{(lndStC)}
\end{Highlighting}
\end{Shaded}

As with the spatial subset function, the above code is extremely terse.
The aggregate function here does three things: 1) identifies which
stations are in which London borough; 2) uses this information to
perform a function on the output, in this case \texttt{length}, which
simply means ``count'' in this context; and 3) creates a new spatial
object equivalent to \texttt{lnd} but with updated attribute data to
reflect the results of the spatial aggregation. The results, with a
legend and colours added, are presented in Fig !!! below.

\begin{figure}[htbp]
\centering
\includegraphics{figure/nStations.png}
\caption{Number of stations in London boroughs}
\end{figure}

As with any spatial attribute data stored as an \texttt{sp} object, we
can look at the attributes of the point data using the \texttt{@}
symbol:

\begin{Shaded}
\begin{Highlighting}[]
\KeywordTok{head}\NormalTok{(lnd.stations@data)}
\end{Highlighting}
\end{Shaded}

\begin{verbatim}
##    CODE          LEGEND FILE_NAME NUMBER                   NAME MICE
## 91 5520 Railway Station  gb_south  17607        Belmont Station   19
## 92 5520 Railway Station  gb_south  17608  Woodmansterne Station    5
## 93 5520 Railway Station  gb_south  17609 Coulsdon South Station   11
## 94 5520 Railway Station  gb_south  17610        Smitham Station   14
## 95 5520 Railway Station  gb_south  17611         Kenley Station   11
## 96 5520 Railway Station  gb_south  17612        Reedham Station    8
\end{verbatim}

In this case we have three potentially interesting variables:
``LEGEND'', telling us what the point is, ``NAME'', and ``MICE'', which
represents the number of mice sightings reported by the public at that
point (this is a fictional variable). To illustrate the power of the
\texttt{aggregate} function, let us use it to find the average number of
mices spotted in transport points in each London borough, and the
standard deviation:

\begin{Shaded}
\begin{Highlighting}[]
\NormalTok{lndAvMice <-}\StringTok{ }\KeywordTok{aggregate}\NormalTok{(lnd.stations[}\StringTok{"MICE"}\NormalTok{], }\DataTypeTok{by =} \NormalTok{lnd, }\DataTypeTok{FUN =} \NormalTok{mean)}
\KeywordTok{summary}\NormalTok{(lndAvMice)}
\NormalTok{lndSdMice <-}\StringTok{ }\KeywordTok{aggregate}\NormalTok{(lnd.stations[}\StringTok{"MICE"}\NormalTok{], }\DataTypeTok{by =} \NormalTok{lnd, }\DataTypeTok{FUN =} \NormalTok{sd)}
\KeywordTok{summary}\NormalTok{(lndSdMice)}
\end{Highlighting}
\end{Shaded}

\subsubsection{Clipping}

\section{References}

Bivand, R., \& Gebhardt, A. (2000). Implementing functions for spatial
statistical analysis using the language. Journal of Geographical
Systems, 2(3), 307--317.

Bivand, R. S., Pebesma, E. J., \& Rubio, V. G. (2008). Applied spatial
data: analysis with R. Springer.

Burrough, P. A. \& McDonnell, R. A. (1998). Principals of Geographic
Information Systems (revised edition). Clarendon Press, Oxford.

Goodchild, M. F. (2007). Citizens as sensors: the world of volunteered
geography. GeoJournal, 69(4), 211--221.

Harris, R. (2012). A Short Introduction to R.
\href{http://www.social-statistics.org/}{social-statistics.org}.

Kabacoff, R. (2011). R in Action. Manning Publications Co.

Krygier, J. Wood, D. 2011. Making Maps: A Visual Guide to Map Design for
GIS (2nd Ed.). New York: The Guildford Press.

Longley, P., Goodchild, M. F., Maguire, D. J., \& Rhind, D. W. (2005).
Geographic information systems and science. John Wiley \& Sons.

Monkhouse, F.J. and Wilkinson, H. R. (1973). Maps and Diagrams Their
Compilation and Construction (3rd Edition, reprinted with revisions).
London: Methuen \& Co Ltd.

Ramsey, P., \& Dubovsky, D. (2013). Geospatial Software's Open Future.
GeoInformatics, 16(4).

Sherman, G. (2008). Desktop GIS: Mapping the Planet with Open Source
Tools. Pragmatic Bookshelf.

Torfs and Brauer (2012). A (very) short Introduction to R. The
Comprehensive R Archive Network.

Venables, W. N., Smith, D. M., \& Team, R. D. C. (2013). An introduction
to R. The Comprehensive R Archive Network (CRAN). Retrieved from
http://cran.ma.imperial.ac.uk/doc/manuals/r-devel/R-intro.pdf .

Wickham, H. (2009). ggplot2: elegant graphics for data analysis.
Springer.

Wickham, H. (2010). A Layered Grammar of Graphics. American Statistical
Association, Institute of Mathematics Statistics and Interface
Foundation of North America Journal of Computational and Graphical
Statistics. 19, 1: 3-28.

\section{Endnotes}

\begin{enumerate}
\def\labelenumi{\arabic{enumi}.}
\item
  R's name originates from the creators of R, Ross Ihaka and Robert
  Gentleman. R is an open source implementation of the statistical
  programming language S, so its name is also a play on words that makes
  implicit reference to this.
\item
  R is notoriously difficult to search for on major search engines, as
  it is such a common letter with many other uses beyond the name of a
  statistical programming language. This should not be a deterrent, as R
  has a wealth of excellent online resources. To overcome the issue, you
  can either be more specific with the search term (e.g. ``R spatial
  statistics'') or use an R specific search engine such as
  \href{http://www.rseek.org/}{rseek.org}. You can also search of online
  help \emph{from within R} using the command \texttt{RSiteSearch}. E.g.
  \texttt{RSiteSearch("spatial statistics")}. Experiment and see which
  you prefer!
\end{enumerate}

\end{document}
